\documentclass[t]{beamer}

\usetheme{CambridgeUS}

\title{Introduction to YUI 3}
\author{Jeff Craig}
\date{May 13, 2010}

\begin{document}
\maketitle

\section{About Me}
\begin{frame}
 \frametitle{Who am I?}
 Software Developer at Washington State University

 \bigskip
 Contact:
 \begin{itemize}
  \item foxxtrot@foxxtrot.net
  \item http://blog.foxxtrot.net/
  \item twitter.com/foxxtrot
  \item github.com/foxxtrot
 \end{itemize}
\end{frame}

\section{YUI}
\subsection{What}
\begin{frame}
 \frametitle{What is YUI?}
 \begin{itemize}
  \item Housed and Developed at Yahoo!
  \item YUI2 Released in 2006, still actively supported
  \item YUI3 launched late 2009
  \item Used across most Yahoo! properties, new development is in YUI3
  \item Designed to be scalable, fast, secure, and modular
 \end{itemize}
\end{frame}

\begin{frame}
 \frametitle{YUI3 Structure}
 \begin{itemize}
  \item \alt<2>{Core
   \begin{itemize}
    \item Lang, UA, Queue, Object, Get, Array, Node, Event
   \end{itemize}}{Core}
  \item \alt<3>{Component Infrastructure
   \begin{itemize}
    \item Base, Attribute, Plugin, Widget
   \end{itemize}}{Component Infrastructure}
  \item \alt<4>{Utilities
   \begin{itemize}
    \item Animation, Cache, Cookie, DataSchema, IO, JSON, ImageLoader, Internationalization, etc.
   \end{itemize}}{Utilities}
  \item \alt<5>{Widgets
   \begin{itemize}
    \item Overlay, Slider, TabView, GridView
   \end{itemize}}{Widgets}
  \item \alt<6>{Developer Tools
   \begin{itemize}
    \item Console, Profiler, Test
   \end{itemize}}{Developer Tools}
 \end{itemize}
\end{frame}

\subsection{Why?}
\begin{frame}
 \frametitle{Why YUI3?}
 \begin{itemize}
  \item Modular: Load only the code you need.
  \item Flexible: Base functionality provides flexible Attribute and Plugin systems
  \item Complete: Tons of utilities available now, and widgets are coming
  \item Fast: Demonstrable faster at common tasks, and fast enough for one of the worlds largest websites.
 \end{itemize}
\end{frame}

\subsection{Why Not?}
\begin{frame}
 \frametitle{Why not YUI3?}
 \begin{itemize}
  \item Your existing codebase works
  \item Many widgets haven't been ported yet.
  \item<2-> Some will not be.
 \end{itemize}
\end{frame}

\subsection{SimpleYUI}
\begin{frame}
 \frametitle{SimpleYUI}
 \begin{itemize}
  \item Introduced with YUI 3.2
  \item Eliminates the Sandbox
  \item Fastest way to get started with YUI3
  \item Provides Node/Events, Transitions, IO
 \end{itemize}
\end{frame}

\begin{frame}
 \frametitle{Why not use SimpleYUI?}
 \begin{itemize}
  \item There are performance benefits to using the module pattern
  \item There is safety in the sandbox
  \item You have more control in standard YUI
 \end{itemize}
\end{frame}

\subsection{Community}
\begin{frame}
 \frametitle{YUI3 and the Community}
 With YUI3, the team refocused on open-source. They launched YUI3 with a public bug tracker and forums, and put the source up on GitHub.

\bigskip
 In October 2009, the Gallery launched, providing a mechanism for modules to be shared easily outside of the core of YUI, including offering hosting on the Yahoo! CDN for modules, and easy inclusion within YUI3 projects.

\bigskip
 In early 2010, the YUI Team began hosting "YUI Open Hours" a periodic conference call.
\end{frame}

\subsection{Resources}
\begin{frame}
 \frametitle{YUI Resources}
 \begin{itemize}
  \item http://yuilibrary.com/
  \item http://developer.yahoo.com/yui/3/
  \item http://github.com/yui/
  \item http://yuiblog.com/
  \item http://twitter.com/miraglia/yui/members
 \end{itemize}
\end{frame}

\subsection{Standard Utilities}
\begin{frame}[fragile]
 \frametitle{Node}
 \begin{verbatim}
Y.one('#nav');
Y.all('#nav li');
 \end{verbatim}
 \begin{itemize}
  \item Y.one returns a 'Node'
  \item Y.all returns a 'NodeList'
 \end{itemize}
\end{frame}

\begin{frame}[fragile]
 \frametitle{Node Methods}
 \begin{verbatim}
var input = Y.one('input[type=text]');
input.addClass('hide'); // Add the 'hide' class to the node
input.removeClass('hide');
input.toggleClass('hide');
input.get('value');
input.getStyle('display');

input.test('.hide'); // Tests if the node matches a selctor
 \end{verbatim}
\end{frame}

\begin{frame}[fragile]
 \frametitle{NodeList Methods}
 \begin{verbatim}
var items = Y.one('li');

items.filter('.hidden');

items.even().addClass('even');
items.odd().addClass('odd');

items.item(4); // NOT an array-like object
items.size();

items.each(function(item, index, list) {
    // Treat this kind of like a for loop.
});
 \end{verbatim}
\end{frame}

\begin{frame}[fragile]
 \frametitle{Events}
 \begin{verbatim}
Y.on('click', handler, '#actionButton');
handler = function(e) {
    // e is a DOM Event Facade
    // Same betwen all browsers
    e.halt(); // Stop propogation and default action
    e.preventDefault();
    e.stopPropogation();
}
 \end{verbatim}
 \begin{itemize}
  \item Same syntax as custom events
  \item Support for touch events
 \end{itemize}
\end{frame}

\begin{frame}[fragile]
 \frametitle{Event Delegation}
 \begin{verbatim}
Y.delegate('click', handler, '#container', selector);
 \end{verbatim}
 \begin{itemize}
  \item Assign one event, on the container, but only call the handler on children that match the selector.
  \item<2-> Doesn't fix non-bubbling events, like change-events in Internet Explorer 
 \end{itemize}
\end{frame}

\subsection{AJAX}
\begin{frame}[fragile]
 \frametitle{Server Calls}
 \begin{verbatim}
Y.io("/path/to/request", {
        method: 'GET',
        data: 'field1=value&field2=3',
        on: {
            start: handler,
            complete: handler,
            success: handler,
            failure: handler,
            end: handler
        }
        sync: false,
        timeout: 2000
    });
 \end{verbatim}
\end{frame}

\begin{frame}[fragile]
 \frametitle{JSON}
\begin{verbatim}
Y.JSON.stringify({
    name: "Jeff Craig",
    nick: "foxxtrot",
    session: "Intro to YUI3"
});


Y.JSON.parse("{'event': 'Palouse Code Camp', 
               'date': '2010.10.30'}");

\end{verbatim}
\end{frame}

\begin{frame}[fragile]
 \frametitle{Transistions}
\begin{verbatim}
Y.one('#demo').transition({
  duration: 1, // seconds
  easing: 'ease-out',
  height: 0,
  width: 0,
  left: '150px',
  top: '100px',
  opacity: 0
}, function() { Y.one('#demo').destroy(true); } );
\end{verbatim}
\end{frame}

\subsection{End Notes}
\begin{frame}
 \frametitle{Links}
 YUI:
 \begin{itemize}
  \item http://yuilibrary.com/
  \item http://developer.yahoo.com/yui/3/
  \item http://github.com/yui/
  \item http://yuiblog.com/
  \item http://twitter.com/miraglia/yui/members
 \end{itemize}

 Me:
 \begin{itemize}
  \item foxxtrot@foxxtrot.net
  \item http://blog.foxxtrot.net/
  \item twitter.com/foxxtrot
  \item github.com/foxxtrot
 \end{itemize}
\end{frame}
\end{document}
